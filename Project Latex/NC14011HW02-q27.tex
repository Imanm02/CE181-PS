سوال تئوری ۱۶

ماتریس
$A$
مربوط به این سوال به شکل زیر خواهد بود
\\
\\
$$
\begin{bmatrix}
1 & 1 & 0 & 0 \\
1 & 1 & 0 & 0 \\
0 & 0 & 1 & 1 \\
0 & 0 & 1 & 1 \\ 
\end{bmatrix}
$$
\\
\\
همجنین ماتریس 
$W$
به شکل زیر خواهد بود:
\\
\\
$$
\begin{bmatrix}
p & p & q & q \\
p & p & q & q \\
q & q & p & p \\
q & q & p & p \\ 
\end{bmatrix}
$$
\\
\\
$$|W - I\lambda| = 0$$
\\


برای محاسبه‌ی مقدار ویژه، باید از ماتریس داده شده، لاندا برابر از ماتریس همانی را کم کرده، از ماتریس حاصل دترمینان بگیریم و آن را مساوی صفر قرار دهیم. با حل کردن معادله، مقادیر مختلف لاندا بدست می‌آید که مقادیر ویژه‌ی ما هستند
\\
بنابرین داریم:
\\
\\
$$
\begin{bmatrix}
p - \lambda & p & q & q \\
p & p - \lambda & q & q \\
q & q & p - \lambda & p \\
q & q & p & p -\lambda \\ 
\end{bmatrix}
$$
\\
\\
می‌دانیم در محاسبه‌ی ماتریس، برای سهولت کار می‌توانیم ضریبی از یکی از سطرها را از سطری دیگر کم کنیم. پس با استناد به این، دترمینان ماتریس بالا با این ماتریس برابر خواهد بود:
\\
\\
$$
\begin{bmatrix}
-\lambda & \lambda & 0 & 0 \\
p & p - \lambda & q & q \\
q & q & p - \lambda & p \\
0 & 0 & \lambda & -\lambda \\ 
\end{bmatrix}
$$
\\
\\
دترمینان حاصل خواهد بود:
\\
$$\lambda^2(\lambda^2 - 4p\lambda + 4(p^2 - q^2))$$
\\
با برابر صفر قرار دادن عبارت بالا، جواب‌های ممکن برای لاندا بدین شکل خواهند بود:
\\

$(0, 0, 2p + 2q, 2p - 2q)$
\\
با داشتن لانداها، بردار ویژه‌ها را محاسبه خواهیم کرد:
\\
$$\frac{1}{\sqrt{4}}
\begin{bmatrix}
 1 \\
-1 \\
 1 \\
-1 \\ 
\end{bmatrix}$$

\\