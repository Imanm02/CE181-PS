سوال تئوری ۲۸

با توجه به پرسش قبل و بزرگ بودن 
$n$
طبق قضیه حد مرکزی میتوانیم متغیر تصادفی
$deg_i$
که درجه راس 
$i$
است را با توزیع نورمال تخمین بزنیم.

$$Var[deg_i]=(n-1)p(1-p) \simeq 0.16$$

پس می‌توانیم فرض کنیم احتمال اینکه راس درجه بیشتر از ۲ داشته باشیم صفر است.
و اینکه احتمال اینکه میانگین درجات زیاد باشد نیز نزدیک صفر است، پس از آن هم صرف نظر می‌کنیم.
در نتیجه احتمال اینکه یک نفر هم‌رنگ نباشد برابر احتمال این است که این راس هیچ یالی نداشته باشد.
و این مقدار برابر
$(1-p)^{n-1}$
است.
در نتیجه احتمال هم‌رنگ بودن یک نفر برابر
$1-(1-p)^{n-1} \simeq 0.147$
است و امیدریاضی تعداد هم‌رنگ‌ها برابر 
$147$
می‌شود.