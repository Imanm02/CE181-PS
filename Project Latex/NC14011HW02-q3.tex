سوال تئوری ۳   

فرض کنید M ژانر محتلف فیلم داریم. اگر دو نفر به ژانر ۲، ۱- i ام علاقه داشته باشند، به احتمال i با یکدیگر هم‌سلیقه‌اند، حال فرض کنید دو شخص خاص هر کدام به چند ژانر مختلف علاقه دارند. اگر فرض کنیم علاقه به pi ژانرهای مختلف از هم مستقلند، احتمال اینکه این دو نفر با یکدیگر هم سلیقه باشند چقدر است؟

چرا اگر دو نفر در جوامع خاص زیادی عضو باشند احتمال اینکه با هم در ارتباط باشند بیشتر می شود؟ آیا می‌توانید این مسئله را
به صورت شهودی هم توجیه کنید؟ به نظر شما این مدل چه نقصی در مدل کردن رابطه‌ی افراد و گروه ها دارد؟ شکل ۳ را در این خصوص ببینید.

احتمال هم‌سلیقه بودن با فرض علاقه‌به ژانرهای مجموعه‌ی A :



$$
1-\prod _{i\in A}\left( 1-p_{i}\right)
$$


زیرا هر چه در جوامع بیشتری باشند، تعداد جملات پای در فرمول بالا بیشتر می‌شود و چون هر جمله کمتر از یک است، پس در نهایت احتمال  هم‌سلیقگی بیشتر می‌شود.
تعداد ژانرها محدود است و هر چه تعداد بیشتری را دوست داشته باشید در ارتباط با افراد بیشتری که آن را دوست دارند قرار می‌گیرید و احتمال اینکه با فردی هم‌سلیقه شوید بیشتر خواهد بود بود. پس به طور شهودی نیز تطابق دارد.
- نقص‌ها
    - تعلق به یک گروه صفر و یکی است و وزن ندارد درصورتی که وزنی بهتر توصیف می‌کند
    - حالات بین خوشه‌ای به طور بهینه مدل نشده‌اند، زیرا ممکن است یک ترکیب از تعدادی خوشه خودش یک خوشه‌ی دارای اصالت باشد.
    - ممکن است خوشه‌ها بر اساس ژانر فیلم نباشند و معیارهای دیگری مثل کارگردان و بازیگران و کمپانی و ... نیز در خوشه‌ها باشند.