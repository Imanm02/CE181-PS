سوال تئوری ۱

اگر جامعه را بتوان به خوشه‌های مجزا از افرادی با سلیقه‌ی یکسان افراز کرد، به طوری که افرادِ یک خوشه با افرادِ خوشه‌ی دیگر هیچ اشتراکی نداشته باشند، آن‌گاه ماتریس مجاورت این افراد به چه صورت خواهد بود؟

اولا که ماتریس متقارن است زیرا که ‍ روابط دوطرفه هستند. ثانیا به علت اینکه همه‌ی افراد یک خوشه با هم در ارتباط هستند و یال دارند، می‌توان ادعا کرد که یک جایگشت سطری ستونی از ماتریس مجاورت وجود دارد به گونه‌ای روی قطر آن چندین ماتریس مربعی کاملا یک باشند و مابقی ماتریس صفر. مثل ماتریس جوردن ولی همه‌ی درایه‌های بلوک‌ها یک هستند و نه صفر.
